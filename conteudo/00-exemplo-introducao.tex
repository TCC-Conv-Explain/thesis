%!TeX root=../tese.tex
%("dica" para o editor de texto: este arquivo é parte de um documento maior)
% para saber mais: https://tex.stackexchange.com/q/78101

%% ------------------------------------------------------------------------- %%

% "\chapter" cria um capítulo com número e o coloca no sumário; "\chapter*"
% cria um capítulo sem número e não o coloca no sumário. A introdução não
% deve ser numerada, mas deve aparecer no sumário. Por conta disso, este
% modelo define o comando "\chapter**".

% Falar na introdução:

% começar explicando o crescimento de redes neurais, ia e inteligência artificial 
% nos últimos tempos, da importância de redes convolucionais e o quão perigoso 
% elas podem ser por conta de sua natureza "desconhecida", já que são modelos 
% explicáveis e não interpretáveis. 

% Explicar a importância de se justificar decisões de modelos em situações do mundo real 
% e como pode ser prejudicial a deficiência disso 

% Como fazer citação: \citet{(ccódigo de citação)} para citação direta
% e \citep{(ccódigo de citação)} para citação por entre parênteses.

% Use \emph para enfatizar texto e \textit para texto em itálico.

\chapter**{Introduction}
\label{chap:introduction}

\enlargethispage{.5\baselineskip}

\lipsum[1-2]

\section**{Other Section Example}
\label{sec:consideracoes_preliminares}

\lipsum[3-4]

\section**{Next Section example}

\lipsum[5-7]
