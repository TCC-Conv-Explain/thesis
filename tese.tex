\documentclass[a4paper,12pt,twoside,brazilian,english]{book}
\usepackage{imegoodies}
\usepackage[thesis]{imelooks}
\usepackage{lipsum}

\graphicspath{{figuras/},{fig/},{logos/},{img/},{images/},{imagens/}}

\addbibresource{bibliografia.bib}

\title{Título do trabalho}[um subtítulo]
\translatedtitle{Title of the document}[a subtitle]

\author[fem]{Nome Completo}

\def\profa{Prof\kern.02em.\kern-.07emª\kern.07em}
\def\dra{Dr\kern-.04em.\kern-.11emª\kern.07em}

\orientador[fem]{\profa{} \dra{} Fulana de Tal}

\banca{
    \profa{} \dra{} Fulana de Tal (orientadora) -- IME-USP [sem ponto final],
    % Em inglês, não há o "ª"
    %Prof. Dr. Fulana de Tal (advisor) -- IME-USP [sem ponto final],
    Prof. Dr. Ciclano de Tal -- IME-USP [sem ponto final],
    \profa{} \dra{} Convidada de Tal -- IMPA [sem ponto final],
}

\tipotese{
  %mestrado,
  %doutorado,
  tcc,
  %definitiva, % É a versão para defesa ou a versão definitiva?
  %quali, % É qualificação?
  programa={Ciência da Computação},
}

\defesa{
  local={São Paulo},
  data=2017-08-10, % YYYY-MM-DD
}

\direitos{CC-BY}

\fichacatalografica{}  % n sei pra que isso serve. Por enquanto ...


\begin{document}

%%%%%%%%%%%%%%%%%%%%%%%%%%% CAPA E PÁGINAS INICIAIS %%%%%%%%%%%%%%%%%%%%%%%%%%%%

\frontmatter

\pagestyle{plain}

\onehalfspacing % Espaçamento 1,5 na capa e páginas iniciais

\maketitle % capa e folha de rosto

%%%%%%%%%%%%%%%% DEDICATÓRIA, AGRADECIMENTOS, RESUMO/ABSTRACT %%%%%%%%%%%%%%%%%%

\begin{dedicatoria}
    Esta seção é opcional e fica numa página separada; ela pode ser usada para
    uma dedicatória ou epígrafe.
\end{dedicatoria}


% Reinicia o contador de páginas (a próxima página recebe o número "i") para
% que a página da dedicatória não seja contada.
\pagenumbering{roman}

\chapter*{Agradecimentos}
\epigrafe{Do. Or do not. There is no try.}{Mestre Yoda}

Texto texto texto texto texto texto texto texto texto texto texto texto texto
texto texto texto texto texto texto texto texto texto texto texto texto texto
texto texto texto texto texto texto texto texto texto texto texto texto texto
texto texto texto texto. Texto opcional.

%!TeX root=../tese.tex
%("dica" para o editor de texto: este arquivo é parte de um documento maior)
% para saber mais: https://tex.stackexchange.com/q/78101

% As palavras-chave são obrigatórias, em português e em inglês, e devem ser
% definidas antes do resumo/abstract. Acrescente quantas forem necessárias.
\palavraschave{IA, Aprendizado de Máquina, IA Explicável, XAI, Visualização de Características, GradCam, LIME}

\keywords{AI, Machine Learning, Explainable AI, XAI, Feature visualization, GradCam, LIME}

% O resumo é obrigatório, em português e inglês. Estes comandos também
% geram automaticamente a referência para o próprio documento, conforme
% as normas sugeridas da USP.
\resumo{
Com a ascensão do uso de Aprendizado de Máquina para problemas de Visão Computacional, o uso de Redes Neurais Convolucionais (CNNs) se mostrou uma peça fundamental para a criação de modelos estado da arte em tarefas como classificação, detecção de objetos e até mesmo segmentação.
No entanto,em muitos casos, a simples obtenção do resultado de uma predição não é suficiente, sendo necessária uma justificativa para as decisões do modelo.
Utilizando IA Explicável (XAI), podemos encontrar possíveis explicações para predições de modelos complexos como Redes Convolucionais. 
Nesse trabalho, foram estudadas diversas técnicas de Explicabilidade aplicadas a CNNs, utilizando de técnicas como GradCam e Visualização de Características. 
Além disso, foram conduzidos experimentos com cada técnica abordada visando avaliar a eficácia na interpretação dos modelos de Visão Computacional.
}

\abstract{
With the rise of Machine Learning in Computer Vision problems, the use of Convolutional Neural Networks (CNNs) has proven to be a fundamental component in developing state-of-the-art models for tasks such as classification, object detection, and even segmentation.
However, in many cases, simply obtaining the result of a prediction is not sufficient; a justification for the model's decisions is necessary.
By employing Explainable AI (XAI), it is possible to identify potential explanations for the predictions of complex models such as Convolutional Neural Networks.
In this study, various explainability techniques applied to CNNs were analyzed, utilizing methods such as Grad-CAM and Feature Visualization.
Additionally, experiments were conducted with each technique to assess their effectiveness in interpreting Computer Vision models.
}


%%%%%%%%%%%%%%%%%%%%%%%%%%% LISTAS DE FIGURAS ETC. %%%%%%%%%%%%%%%%%%%%%%%%%%%%%

% Settings importantes definidos pelo pessoal do github que administra os templates
\cleardoublepage

\newcommand\disablenewpage[1]{{\let\clearpage\par\let\cleardoublepage\par #1}}

\bgroup
\raggedbottom

%%%%% Listas criadas manualmente

%\chapter*{Lista de abreviaturas}
\disablenewpage{\chapter*{Lista de abreviaturas}}

\begin{tabular}{rl}
   CFT & Transformada contínua de Fourier (\emph{Continuous Fourier Transform})\\
   DFT & Transformada discreta de Fourier (\emph{Discrete Fourier Transform})\\
  EIIP & Potencial de interação elétron-íon (\emph{Electron-Ion Interaction Potentials})\\
  STFT & Transformada de Fourier de tempo reduzido (\emph{Short-Time Fourier Transform})\\
  ABNT & Associação Brasileira de Normas Técnicas\\
   URL & Localizador Uniforme de Recursos (\emph{Uniform Resource Locator})\\
   IME & Instituto de Matemática e Estatística\\
   USP & Universidade de São Paulo
\end{tabular}

%\chapter*{Lista de símbolos}
\disablenewpage{\chapter*{Lista de símbolos}}

\begin{tabular}{rl}
  $\omega$ & Frequência angular\\
    $\psi$ & Função de análise \emph{wavelet}\\
    $\Psi$ & Transformada de Fourier de $\psi$\\
\end{tabular}

% Quebra de página manual
\clearpage

%%%%% Listas criadas automaticamente

% Você pode escolher se quer ou não permitir a quebra de página
%\listoffigures
\disablenewpage{\listoffigures}

% Você pode escolher se quer ou não permitir a quebra de página
%\listoftables
\disablenewpage{\listoftables}

% Esta lista é criada "automaticamente" pela package float quando
% definimos o novo tipo de float "program" (em utils.tex)
% Você pode escolher se quer ou não permitir a quebra de página
%\listof{program}{\programlistname}
\disablenewpage{\listof{program}{\programlistname}}

% Sumário (obrigatório)
\tableofcontents

\egroup % Final de "raggedbottom"

%%%%%%%%%%%%%%%%%%%%%%%%%%%%%%%% CAPÍTULOS %%%%%%%%%%%%%%%%%%%%%%%%%%%%%%%%%%%%%

\mainmatter

\pagestyle{mainmatter}

% Espaçamento simples
\singlespacing

% A introdução não tem número de capítulo, então os cabeçalhos também não
\pagestyle{unnumberedchapter}

% Capitulos:
%!TeX root=../tese.tex
%("dica" para o editor de texto: este arquivo é parte de um documento maior)
% para saber mais: https://tex.stackexchange.com/q/78101

%% ------------------------------------------------------------------------- %%

% "\chapter" cria um capítulo com número e o coloca no sumário; "\chapter*"
% cria um capítulo sem número e não o coloca no sumário. A introdução não
% deve ser numerada, mas deve aparecer no sumário. Por conta disso, este
% modelo define o comando "\chapter**".

% Falar na introdução:

% começar explicando o crescimento de redes neurais, ia e inteligência artificial 
% nos últimos tempos, da importância de redes convolucionais e o quão perigoso 
% elas podem ser por conta de sua natureza "desconhecida", já que são modelos 
% explicáveis e não interpretáveis. 

% Explicar a importância de se justificar decisões de modelos em situações do mundo real 
% e como pode ser prejudicial a deficiência disso 

% Como fazer citação: \citet{(ccódigo de citação)} para citação direta
% e \citep{(ccódigo de citação)} para citação por entre parênteses.

% Use \emph para enfatizar texto e \textit para texto em itálico.

\chapter**{Introduction}
\label{chap:introduction}

\enlargethispage{.5\baselineskip}

\lipsum[1-2]

\section**{Other Section Example}
\label{sec:consideracoes_preliminares}

\lipsum[3-4]

\section**{Next Section example}

\lipsum[5-7]



%%%%%%%%%%%%%%%%%%%%%%%%%%%% APÊNDICES E ANEXOS %%%%%%%%%%%%%%%%%%%%%%%%%%%%%%%%

%%%% Apêndices %%%%

\cleardoublepage

\pagestyle{appendix}

\appendix

% \addappheadtotoc acrescenta a palavra "Apêndice" ao sumário; se
% só há apêndices, sem anexos, provavelmente não é necessário.
\addappheadtotoc

\input{conteudo/apendice-exemplo-faq}
\par

%%%% Anexos %%%%

\cleardoublepage

\pagestyle{appendix} % repete o anterior, caso você não use apêndices

\annex

% \addappheadtotoc acrescenta a palavra "Anexo" ao sumário; se
% só há anexos, sem apêndices, provavelmente não é necessário.
\addappheadtotoc

% \input{conteudo/anexo-exemplo-imegoodies}  % crashando aqui
\par
% \input{conteudo/anexo-exemplo-pseudocodigo}  % crashando aqui também
\par

%%%%%%%%%%%%%%% SEÇÕES FINAIS (BIBLIOGRAFIA E ÍNDICE REMISSIVO) %%%%%%%%%%%%%%%%

% O comando backmatter desabilita a numeração de capítulos.
\backmatter

\pagestyle{backmatter}

% Espaço adicional no sumário antes das referências / índice remissivo
\addtocontents{toc}{\vspace{2\baselineskip plus .5\baselineskip minus .5\baselineskip}}

% A bibliografia é obrigatória

\printbibliography[
  title=\refname\label{sec:bib}, % "Referências", recomendado pela ABNT
  %title=\bibname\label{sec:bib}, % "Bibliografia"
  heading=bibintoc, % Inclui a bibliografia no sumário
]

\end{document}