\documentclass[a4paper,12pt,twoside,brazilian,english]{book}
\usepackage{imegoodies}
\usepackage[thesis]{imelooks}

\graphicspath{{figuras/},{fig/},{logos/},{img/},{images/},{imagens/}}

\title{Título do trabalho}[um subtítulo]
\translatedtitle{Title of the document}[a subtitle]

\author[mas]{Antonio Fernando Silva e Cruz Filho \\ João Gabriel Andrade de Araujo Josephik}


\def\profa{Prof\kern.02em.\kern-.07emª\kern.07em}
\def\dra{Dr\kern-.04em.\kern-.11emª\kern.07em}

\orientador[fem]{Prof. Nina S. T. Hirata}

\banca{
    \profa{} \dra{} Fulana de Tal (orientadora) -- IME-USP [sem ponto final],
    % Em inglês, não há o "ª"
    %Prof. Dr. Fulana de Tal (advisor) -- IME-USP [sem ponto final],
    Prof. Dr. Ciclano de Tal -- IME-USP [sem ponto final],
    \profa{} \dra{} Convidada de Tal -- IMPA [sem ponto final],
}

\tipotese{
  %mestrado,
  %doutorado,
  tcc,
  %definitiva, % É a versão para defesa ou a versão definitiva?
  %quali, % É qualificação?
  programa={Ciência da Computação},
}

\defesa{
  local={São Paulo},
  data=2017-08-10, % YYYY-MM-DD
}

\direitos{CC-BY}

\fichacatalografica{}  % n sei pra que isso serve. Por enquanto ...


\begin{document}

%%%%%%%%%%%%%%%%%%%%%%%%%%% CAPA E PÁGINAS INICIAIS %%%%%%%%%%%%%%%%%%%%%%%%%%%%

\frontmatter

\pagestyle{plain}

\onehalfspacing % Espaçamento 1,5 na capa e páginas iniciais

\maketitle % capa e folha de rosto

%%%%%%%%%%%%%%%% DEDICATÓRIA, AGRADECIMENTOS, RESUMO/ABSTRACT %%%%%%%%%%%%%%%%%%


% Reinicia o contador de páginas (a próxima página recebe o número "i") para
% que a página da dedicatória não seja contada.
\pagenumbering{roman}

\chapter*{Agradecimentos}
\epigrafe{Do. Or do not. There is no try.}{Mestre Yoda}

Texto texto texto texto texto texto texto texto texto texto texto texto texto
texto texto texto texto texto texto texto texto texto texto texto texto texto
texto texto texto texto texto texto texto texto texto texto texto texto texto
texto texto texto texto. Texto opcional.

%!TeX root=../tese.tex
%("dica" para o editor de texto: este arquivo é parte de um documento maior)
% para saber mais: https://tex.stackexchange.com/q/78101

% As palavras-chave são obrigatórias, em português e em inglês, e devem ser
% definidas antes do resumo/abstract. Acrescente quantas forem necessárias.
\palavraschave{IA, Aprendizado de Máquina, IA Explicável, XAI, Visualização de Características, GradCam, LIME}

\keywords{AI, Machine Learning, Explainable AI, XAI, Feature visualization, GradCam, LIME}

% O resumo é obrigatório, em português e inglês. Estes comandos também
% geram automaticamente a referência para o próprio documento, conforme
% as normas sugeridas da USP.
\resumo{
Com a ascensão do uso de Aprendizado de Máquina para problemas de Visão Computacional, o uso de Redes Neurais Convolucionais (CNNs) se mostrou uma peça fundamental para a criação de modelos estado da arte em tarefas como classificação, detecção de objetos e até mesmo segmentação.
No entanto,em muitos casos, a simples obtenção do resultado de uma predição não é suficiente, sendo necessária uma justificativa para as decisões do modelo.
Utilizando IA Explicável (XAI), podemos encontrar possíveis explicações para predições de modelos complexos como Redes Convolucionais. 
Nesse trabalho, foram estudadas diversas técnicas de Explicabilidade aplicadas a CNNs, utilizando de técnicas como GradCam e Visualização de Características. 
Além disso, foram conduzidos experimentos com cada técnica abordada visando avaliar a eficácia na interpretação dos modelos de Visão Computacional.
}

\abstract{
With the rise of Machine Learning in Computer Vision problems, the use of Convolutional Neural Networks (CNNs) has proven to be a fundamental component in developing state-of-the-art models for tasks such as classification, object detection, and even segmentation.
However, in many cases, simply obtaining the result of a prediction is not sufficient; a justification for the model's decisions is necessary.
By employing Explainable AI (XAI), it is possible to identify potential explanations for the predictions of complex models such as Convolutional Neural Networks.
In this study, various explainability techniques applied to CNNs were analyzed, utilizing methods such as Grad-CAM and Feature Visualization.
Additionally, experiments were conducted with each technique to assess their effectiveness in interpreting Computer Vision models.
}


%%%%%%%%%%%%%%%%%%%%%%%%%%% LISTAS DE FIGURAS ETC. %%%%%%%%%%%%%%%%%%%%%%%%%%%%%

% Settings importantes definidos pelo pessoal do github que administra os templates
\cleardoublepage

\newcommand\disablenewpage[1]{{\let\clearpage\par\let\cleardoublepage\par #1}}

\bgroup
\raggedbottom

%%%%% Listas criadas manualmente

%\chapter*{Lista de abreviaturas}
\disablenewpage{\chapter*{List of Abbreviations}}

\begin{tabular}{rl}
   MLP & Multilayer Perceptron \\
   CNN & Convolutional Neural Network \\
   Conv & Convolution \\
   IME & Institute of Mathematics and Statistics\\
   USP & University of São Paulo
\end{tabular}

% Quebra de página manual
\clearpage

%%%%% Listas criadas automaticamente

% Você pode escolher se quer ou não permitir a quebra de página
%\listoffigures
\disablenewpage{\listoffigures}

% Você pode escolher se quer ou não permitir a quebra de página
%\listoftables
\disablenewpage{\listoftables}

% Esta lista é criada "automaticamente" pela package float quando
% definimos o novo tipo de float "program" (em utils.tex)
% Você pode escolher se quer ou não permitir a quebra de página
%\listof{program}{\programlistname}
\disablenewpage{\listof{program}{\programlistname}}

% Sumário (obrigatório)
\tableofcontents

\egroup % Final de "raggedbottom"

%%%%%%%%%%%%%%%%%%%%%%%%%%%%%%%% CAPÍTULOS %%%%%%%%%%%%%%%%%%%%%%%%%%%%%%%%%%%%%

\mainmatter

\pagestyle{mainmatter}

% Espaçamento simples
\singlespacing

% A introdução não tem número de capítulo, então os cabeçalhos também não
\pagestyle{unnumberedchapter}

% Capitulos:
% %!TeX root=../tese.tex
%("dica" para o editor de texto: este arquivo é parte de um documento maior)
% para saber mais: https://tex.stackexchange.com/q/78101

%% ------------------------------------------------------------------------- %%

% "\chapter" cria um capítulo com número e o coloca no sumário; "\chapter*"
% cria um capítulo sem número e não o coloca no sumário. A introdução não
% deve ser numerada, mas deve aparecer no sumário. Por conta disso, este
% modelo define o comando "\chapter**".
\chapter**{Introdução}
\label{cap:introducao}

\enlargethispage{.5\baselineskip}

Escrever bem é uma arte que exige muita técnica e dedicação e,
consequentemente, há vários bons livros sobre como escrever uma boa
dissertação ou tese. Um dos trabalhos pioneiros e mais conhecidos nesse
sentido é o livro de Umberto~\citet{eco:09} intitulado \emph{Como se faz
uma tese}; é uma leitura bem interessante mas, como foi escrito em 1977 e
é voltado para trabalhos de graduação na Itália, não se aplica tanto a nós.

Sobre a escrita acadêmica em geral, John Carlis disponibilizou um texto curto
e interessante~\citep{carlis:09} em que advoga a preparação de um único
rascunho da tese antes da versão final. Mais importante que isso, no
entanto, são os vários \textit{insights} dele sobre a escrita acadêmica.
Dois outros bons livros sobre o tema são \emph{The Craft of Research}~\citep{craftresearch}
e \emph{The Dissertation Journey}~\citep{dissertjourney}. Além disso,
a USP tem uma compilação de normas relativas à produção de documentos
acadêmicos~\citep{usp:guidelines} que pode ser utilizada como referência.

Para a escrita de textos especificamente sobre Ciência da Computação, o
livro de Justin Zobel, \emph{Writing for Computer Science}~\citep{zobel:04}
é uma leitura obrigatória. O livro \emph{Metodologia de Pesquisa para
Ciência da Computação} de Raul Sidnei~\citet{waz:09}
também merece uma boa lida. Já para a área de Matemática, dois livros
recomendados são o de Nicholas Higham, \emph{Handbook of Writing for
Mathematical Sciences}~\citep{Higham:98} e o do criador do \TeX{}, Donald
Knuth, juntamente com Tracy Larrabee e Paul Roberts, \emph{Mathematical
Writing}~\citep{Knuth:96}.

Apresentar os resultados de forma simples, clara e completa é uma tarefa que
requer inspiração. Nesse sentido, o livro de Edward~\citet{tufte01:visualDisplay},
\emph{The Visual Display of Quantitative Information}, serve de ajuda na
criação de figuras que permitam entender e interpretar dados/resultados de forma
eficiente.

Além desse material, também vale muito a pena a leitura do trabalho de
Uri \citet{alon09:how}, no qual apresenta-se uma reflexão sobre a utilização
da Lei de Pareto para tentar definir/escolher problemas para as diferentes
fases da vida acadêmica. A direção dos novos passos para a continuidade da
vida acadêmica deveria ser discutida com seu orientador.

\section**{Considerações de estilo}
\label{sec:consideracoes_preliminares}

Normalmente, as citações não devem fazer parte da estrutura sintática da
frase\footnote{E não se deve abusar das notas de rodapé.\index{Notas de rodapé}}.
No entanto, usando referências em algum estilo autor-data (como o estilo
plainnat do \LaTeX{}), é comum que o nome do autor faça parte da frase. Nesses
casos, pode valer a pena mudar o formato da citação para não repetir o nome do
autor; no \LaTeX{}, isso pode ser feito usando os comandos
\textsf{\textbackslash{}citet}, \textsf{\textbackslash{}citep},
\textsf{\textbackslash{}citeyear} etc. documentados no pacote
natbib \citep{natbib}\index{natbib} (esses comandos são compatíveis com biblatex
usando a opção \textsf{natbib=true}, ativada por padrão neste modelo). Em geral,
portanto, as citações devem seguir estes exemplos:

\bgroup
\footnotesize
\begin{verbatim}
Modos de citação:
indesejável: [AF83] introduziu o algoritmo ótimo.
indesejável: (Andrew e Foster, 1983) introduziram o algoritmo ótimo.
certo: Andrew e Foster introduziram o algoritmo ótimo [AF83].
certo: Andrew e Foster introduziram o algoritmo ótimo (Andrew e Foster, 1983).
certo (\citet ou \citeyear): Andrew e Foster (1983) introduziram o algoritmo ótimo.
\end{verbatim}
\egroup

\enlargethispage{.5\baselineskip}

O uso desnecessário de termos em língua estrangeira deve ser evitado. No entanto,
quando isso for necessário, os termos devem aparecer \textit{em itálico}.
\index{Língua estrangeira}
% index permite acrescentar um item no indice remissivo

Uma prática recomendável na escrita de textos é descrever as
legendas\index{Legendas} das figuras e tabelas em forma auto-contida: as
legendas devem ser razoavelmente completas, de modo que o leitor possa entender
a figura sem ler o texto em que a figura ou tabela é citada.\index{Floats}

\section**{Ferramentas bibliográficas}

Embora seja possível pesquisar por material acadêmico na Internet usando
sistemas de busca ``comuns'', existem ferramentas dedicadas, como o
\textsf{Google Scholar}\index{Google Scholar} (\url{scholar.google.com}).
O \textsf{Web of Science}\index{Web of Science} (\url{webofscience.com})
e o \textsf{Scopus}\index{Scopus} (\url{scopus.com}) oferecem recursos
sofisticados e limitam a busca a periódicos com boa reputação acadêmica.
Essas duas plataformas não são gratuitas, mas os alunos da USP têm
acesso a elas através da instituição. Algumas editoras, como a ACM
(\url{dl.acm.org}) e a IEEE (\url{ieeexplore.ieee.org}), também têm
sistemas de busca bibliográfica. Todas essas ferramentas são capazes de
exportar os dados para o formato .bib, usado pelo \LaTeX{} (no Google
Scholar, é preciso ativar a opção correspondente nas preferências).
O sítio \url{liinwww.ira.uka.de/bibliography} também permite buscar e
baixar referências bibliográficas relevantes para a área de computação.

Lamentavelmente, ainda não existe um mecanismo de verificação ou validação
das informações nessas plataformas. Portanto, é fortemente sugerido validar
todas as informações de tal forma que as entradas bib estejam corretas.
De qualquer modo, tome muito cuidado na padronização das referências
bibliográficas: ou considere TODOS os nomes dos autores por extenso, ou TODOS
os nomes dos autores abreviados. Evite misturas inapropriadas.\looseness=-1

Apenas uma parte dos artigos acadêmicos de interesse está disponível livremente
na Internet; os demais são restritos a assinantes. A CAPES assina um grande
volume de publicações e disponibiliza o acesso a elas para diversas
universidades brasileiras, entre elas a USP, através do seu portal de
periódicos (\url{periodicos.capes.gov.br}). Existe uma extensão para os
navegadores Chrome e Firefox (\url{www.infis.ufu.br/capes-periodicos})
que facilita o uso cotidiano do portal.

Para manter um banco de dados organizado sobre artigos e outras fontes
bibliográficas relevantes para sua pesquisa, é altamente recomendável
que você use uma ferramenta como Zotero~(\url{zotero.org})\index{Zotero}
ou Mendeley~(\url{mendeley.com})\index{Mendeley}. Ambas podem exportar
seus dados no formato .bib, compatível com \LaTeX{}.



%%%%%%%%%%%%%%%%%%%%%%%%%%%% APÊNDICES E ANEXOS %%%%%%%%%%%%%%%%%%%%%%%%%%%%%%%%

%%%% Apêndices %%%%

\cleardoublepage

\pagestyle{appendix}

\appendix

% \addappheadtotoc acrescenta a palavra "Apêndice" ao sumário; se
% só há apêndices, sem anexos, provavelmente não é necessário.
\addappheadtotoc

%!TeX root=../tese.tex
%("dica" para o editor de texto: este arquivo é parte de um documento maior)
% para saber mais: https://tex.stackexchange.com/q/78101

\chapter{Perguntas frequentes sobre o modelo}

\begin{itemize}

\item \textbf{Não consigo decorar tantos comandos!}\\
Use a colinha que é distribuída juntamente com este modelo (\url{gitlab.com/ccsl-usp/modelo-latex/raw/main/pre-compilados/colinha.pdf?inline=false}).

\item \textbf{Estou tendo problemas com caracteres acentuados.}\\
Versões modernas de \LaTeX{} usam UTF-8, mas arquivos antigos podem usar outras codificações (como ISO-8859-1, também conhecido como latin1 ou Windows-1252). Nesses casos, use \textsf{\textbackslash{}usepackage[latin1]\{inputenc\}} no preâmbulo do documento. Você também pode representar os caracteres acentuados usando comandos \LaTeX{}: \textsf{\textbackslash\textquotesingle{}a} para á, \textsf{\textbackslash{}c\{c\}} para cedilha etc., independentemente da codificação usada no texto\footnote{Você pode consultar os comandos desse tipo mais comuns em \url{en.wikibooks.org/wiki/LaTeX/Special_Characters}. Observe que a dica sobre o pingo do i \emph{não} é mais válida atualmente; basta usar \textsf{\textbackslash\textquotesingle{}i}.}.

\item \textbf{É possível resumir o nome das seções/capítulos que aparece no topo das páginas e no sumário?}\\
Sim, usando a sintaxe \textsf{\textbackslash{}section[mini-titulo]\{titulo enorme\}}. Isso é especialmente útil nas legendas (\textit{captions}\index{Legendas}) das figuras e tabelas, que muitas vezes são demasiadamente longas para a lista de figuras/tabelas.

\item \textbf{Existe algum programa para gerenciar referências em formato bibtex?}\\
Sim, há vários. Uma opção bem comum é o JabRef; outra é usar Zotero\index{Zotero} ou Mendeley\index{Mendeley} e exportar os dados deles no formato .bib.

\item \textbf{Posso usar pacotes \LaTeX{} adicionais aos sugeridos?}\\
Com certeza! Você pode modificar os arquivos o quanto desejar, o modelo serve só como uma ajuda inicial para o seu trabalho.

\end{itemize}

\par

%%%% Anexos %%%%

\cleardoublepage

\pagestyle{appendix} % repete o anterior, caso você não use apêndices

\annex

% \addappheadtotoc acrescenta a palavra "Anexo" ao sumário; se
% só há anexos, sem apêndices, provavelmente não é necessário.
\addappheadtotoc

% %!TeX root=../tese.tex
%("dica" para o editor de texto: este arquivo é parte de um documento maior)
% para saber mais: https://tex.stackexchange.com/q/78101

\chapter{As packages \pkg{imegoodies} e \pkg{imelooks}}
\label{ann:imegoodlooks}

Este modelo inclui as \textit{packages} \pkg{imegoodies} e \pkg{imelooks},
que você pode querer usar em outros documentos \LaTeX.

\pkg{imegoodies} inclui um grande número de \textit{packages} que são
comumente usadas e bastante úteis. Em geral, você pode incluí-la em seus
documentos sem que isso cause problemas de compatibilidade. Se, no
entanto, algo não funcionar, você pode editar o arquivo para eliminar
a \textit{package} responsável pelo problema se ela não for necessária.
\pkg{imegoodies} ainda inclui vários comentários explicativos sobre as
\textit{packages} carregadas.

\pkg{imelooks} também inclui um grande número de \textit{packages}, mas
estas são relacionadas mais explicitamente à aparência do documento
(fontes, cores, margens etc.). Você também pode utilizá-la em outros
documentos se quiser se aproximar da aparência deste modelo. \pkg{imelooks}
reconhece diversos parâmetros que ativam/desativam aspectos específicos:

\begin{itemize}
  \item \cmd{fonts} carrega as fontes deste modelo (libertinus e
        sourcecodepro), além de outros pequenos ajustes relacionados.
        Esta opção é sempre ativada por padrão; para desativá-la, use
        \cmd{nofonts}

  \item \cmd{spacing} utiliza os espaçamentos definidos neste modelo (margens,
        espaço entre parágrafos, indentação da primeira linha do parágrafo
        etc.). Esta opção é sempre ativada por padrão; para desativá-la, use
        \cmd{nospacing}

  \item \cmd{captions} e \cmd{footnotes} fazem respectivamente as legendas
        (das figuras e tabelas) e as notas de rodapé de acordo com este modelo.
        Estas opções são sempre ativadas por padrão; para desativá-las, use
        \cmd{nocaptions} e \cmd{nofootnotes}

  \item \cmd{autohttp} acrescenta o prefixo \cmd{http://} a URLs criadas
        com \ltxcmd{url} que não incluam o \textit{schema}. Esta opção é
        sempre ativada por padrão; para desativá-la, use \cmd{noautohttp}

  \item \cmd{hidelinks}, \cmd{borderlinks} e \cmd{colorlinks} definem a
        aparência dos hiperlinks. \cmd{hidelinks} faz os hiperlinks sem
        nenhuma formatação especial; \cmd{borderlinks} faz os hiperlinks
        serem envidos por um quadrado colorido (apenas na tela; o quadrado
        não é impresso); \cmd{colorlinks} faz o texto dos hiperlinks ser
        colorido. A opção \cmd{colorlinks} é sempre ativada por padrão

  \item \cmd{biblatex} carrega a \textit{package} \cmd{biblatex} e os
        estilos bibliográficos deste modelo. Esta opção é sempre ativada
        por padrão; para desativá-la, use \cmd{nobiblatex}
  \item \cmd{raggedbib} faz a bibliografia (com \cmd{biblatex}) ser
        formatada com alinhamento à esquerda ao invés de justificado.
        Esta opção é sempre ativada por padrão, exceto quando o estilo
        bibliográfico é \cmd{plainnat-ime} (usado nas teses); para
        desativá-la, use \cmd{noraggedbib}; para ativá-la incondicionalmente,
        use \cmd{raggedbib}
  \item \cmd{bibstyle=?} selectiona um estilo bibliográfico específico.
        O estilo padrão é \cmd{numeric}, exceto em pôsteres e apresentações
        (\cmd{beamer-ime}) e \textit{reports} (\cmd{plainnat-ime})

  \item \cmd{listings} carrega a \textit{package} \cmd{listings} e diversas
        configurações relacionadas usadas neste modelo. Esta opção é
        sempre ativada por padrão; para desativá-la, use \cmd{nolistings}

  \item \cmd{greeny}, \cmd{bluey}, \cmd{sandy} ativam esquemas de cores
        diferentes para pôsteres e apresentações (o padrão é \cmd{bluey})

  \item \cmd{beamer} \textbf{des}ativa algumas \textit{packages} que
        são incompatíveis com a classe \cmd{beamer} (note que as opções
        \cmd{slides} e \cmd{presentation}, discutidas abaixo, já fazem isso)

  \item \cmd{presentation} (ou \cmd{slides}) e \cmd{poster} ativam as
        opções relevantes para, respectivamente, apresentações com
        \cmd{beamer} ou pôsteres com \cmd{tcolorbox}

  \item \cmd{report} ativa as opções relevantes para documentos com
        capítulos (cabeçalhos das páginas, características do sumário etc.)

  \item \cmd{thesis} ativa a opção \cmd{report} e também define o que é
        necessário para a geração da capa das teses de acordo com este modelo

  \item \cmd{resumoabstract} define os comandos \cmd{resumo} e \cmd{abstract}
        de acordo com este modelo. Esta opção é ativada por padrão com
        \cmd{report}; para desativá-la, use \cmd{noresumoabstract}

  \item \cmd{brazilian} verifica se a língua portuguesa está ativa no
        documento e, em caso negativo, gera um erro. Esta opção é
        ativada por padrão com a opção \cmd{thesis}; para desativá-la,
        use \cmd{nobrazilian}
\end{itemize}
  % crashando aqui
\par
% %!TeX root=../tese.tex
%("dica" para o editor de texto: este arquivo é parte de um documento maior)
% para saber mais: https://tex.stackexchange.com/q/78101

\chapter{Código-fonte e pseudocódigo}
\label{ap:pseudocode}

Com a \textit{package} \textsf{listings}, programas podem ser inseridos
diretamente no arquivo, como feito no caso do Programa~\ref{prog:java},
ou importados de um arquivo externo com o comando
\textsf{\textbackslash{}lstinputlisting}, como no caso
do Programa~\ref{prog:mdcinput}.

% O exemplo foi copiado da documentação de algorithmicx
\begin{program}
  \lstinputlisting[
    language=pseudocode,
    style=pseudocode,
    style=wider,
    functions={},
    specialidentifiers={},
  ]
  {conteudo/euclid.psc}

  \caption{Máximo divisor comum (arquivo importado).\label{prog:mdcinput}}
\end{program}

Trechos de código curtos (menores que uma página) podem ou não ser
incluídos como \textit{floats}; trechos longos necessariamente incluem
quebras de página e, portanto, não podem ser \textit{floats}. Com
\textit{floats}, a legenda e as linhas separadoras são colocadas pelo
comando \textsf{\textbackslash{}begin\{program\}}; sem eles, utilize o
ambiente \textsf{programruledcaption} (atenção para a colocação do
comando \textsf{\textbackslash{}label\{\}}, dentro da legenda), como
no Programa~\ref{prog:mdc}\footnote{\textsf{listings} oferece alguns
recursos próprios para a definição de \textit{floats} e legendas, mas
neste modelo não os utilizamos.}:

\begin{programruledcaption}{Máximo divisor comum (em português).\label{prog:mdc}}
  \begin{lstlisting}[
    language={[brazilian]pseudocode},
    style=pseudocode,
    style=wider,
    functions={},
    specialidentifiers={},
  ]
      funcao euclides(a, b) // O máximo divisor comum de \textbf{a} e \textbf{b}
          r := a $\bmod$ b
	  enquanto r != 0 // Atingimos a resposta se \textbf{r} é zero
              a := b
              b := r
              r := a $\bmod$ b
          fim
	  devolva b // O máximo divisor comum é \textbf{b}
      fim
  \end{lstlisting}
\end{programruledcaption}

Além do suporte às várias linguagens incluídas em \textsf{listings},
este modelo traz uma extensão para permitir o uso de pseudocódigo,
útil para a descrição de algoritmos em alto nível. Ela oferece
diversos recursos:

\begin{itemize}

    \item Comentários seguem o padrão de C++ (\lstinline{//} e
          \lstinline{/* ... */}), mas o delimitador é impresso
          como ``$\triangleright$''.

    \item ``:='', ``<>'', ``<='', ``>='' e ``!='' são substituídos
          pelo símbolo matemático adequado.

    \item É possível acrescentar palavras-chave além de ``if'', ``and''
          etc. com a opção ``\textsf{morekeywords=\{pchave1,\linebreak[0]{}pchave2\}}''
          (para um trecho de código específico) ou com o comando
          \textsf{\textbackslash{}lstset\{morekeywords=\linebreak[0]{}\{pchave1,pchave2\}\}}
          (como comando de configuração geral).

    \item É possível usar pequenos trechos de código, como nomes de variáveis,
          dentro de um parágrafo normal com \textsf{\textbackslash{}lstinline\{blah\}}.

    \item ``\$\dots\$'' ativa o modo matemático em qualquer lugar.

    \item Outros comandos \LaTeX{} funcionam apenas em comentários; fora, a
          linguagem simula alguns pré-definidos (\textsf{\textbackslash{}textit\{\}},
          \textsf{\textbackslash{}texttt\{\}} etc.).

    \item O comando \textsf{\textbackslash{}label} também funciona em
          comentários; a referência correspondente (\textsf{\textbackslash{}ref})
          indica o número da linha de código. Se quiser usá-lo numa linha sem
          comentários, use \lstinline{///}~\textsf{\textbackslash{}label\{blah\}};
          ``\lstinline{///}'' funciona como \lstinline{//}, permitindo
          a inserção de comandos \LaTeX{}, mas não imprime o delimitador
          (\ensuremath{\triangleright}).

    \item Para suspender a formatação automática, use \textsf{\textbackslash{}noparse\{blah\}}.

    \item Para forçar a formatação de um texto como função, identificador,
          palavra-chave ou comentário, use \textsf{\textbackslash{}func\{blah\}},
          \textsf{\textbackslash{}id\{blah\}}, \textsf{\textbackslash{}kw\{blah\}} ou
          \textsf{\textbackslash{}comment\{blah\}}.

    \item Palavras-chave dentro de comentários não são formatadas
          automaticamente; se necessário, use \textsf{\textbackslash{}func\{\}},
          \textsf{\textbackslash{}id\{\}} etc. ou comandos \LaTeX{} padrão.

    \item As palavras ``Program'', ``Procedure'' e ``Function'' têm formatação
          especial e fazem a palavra seguinte ser formatada como função.
          Funções em outros lugares \emph{não} são detectadas automaticamente;
          use \textsf{\textbackslash{}func\{\}}, a opção ``\textsf{functions=\{func1,func2\}}''
          ou o comando ``\textsf{\textbackslash{}lstset\{functions=\{func1,func2\}\}}''
          para que elas sejam detectadas.

    \item Além de funções, palavras-chave, strings, comentários e
          identificadores, há ``\textsf{specialidentifiers}''. Você pode
          usá-los com \textsf{\textbackslash{}specialid\{blah\}}, com a opção
          ``\textsf{specialidentifiers=\{id1,id2\}}'' ou com o comando
          ``\textsf{\textbackslash{}lstset\{specialidentifiers=\{id1,id2\}\}}''.

\end{itemize}


  % crashando aqui também
\par

%%%%%%%%%%%%%%% SEÇÕES FINAIS (BIBLIOGRAFIA E ÍNDICE REMISSIVO) %%%%%%%%%%%%%%%%

% O comando backmatter desabilita a numeração de capítulos.
\backmatter

\pagestyle{backmatter}

% Espaço adicional no sumário antes das referências / índice remissivo
\addtocontents{toc}{\vspace{2\baselineskip plus .5\baselineskip minus .5\baselineskip}}

% A bibliografia é obrigatória

\printbibliography[
  title=\refname\label{sec:bib}, % "Referências", recomendado pela ABNT
  %title=\bibname\label{sec:bib}, % "Bibliografia"
  heading=bibintoc, % Inclui a bibliografia no sumário
]

\end{document}